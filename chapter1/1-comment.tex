\documentclass{article}
\usepackage{mathtools} 
\usepackage{fontspec}
\usepackage[UTF8]{ctex}
\usepackage{amsthm}
\usepackage{mdframed}
\usepackage{xcolor}
\usepackage{amssymb}
\usepackage{amsmath}


% 定义新的带灰色背景的说明环境 zremark
\newmdtheoremenv[
  backgroundcolor=gray!10,
  % 边框与背景一致,边框线会消失
  linecolor=gray!10
]{zremark}{说明}

% 通用矩阵命令: \flexmatrix{矩阵名}{元素符号}{行数}{列数}
\newcommand{\flexmatrix}[4]{
  \[
  #1 = \begin{pmatrix}
    #2_{11}     & #2_{12}     & \cdots & #2_{1#4}   \\
    #2_{21}     & #2_{22}     & \cdots & #2_{2#4}   \\
    \vdots      & \vdots      & \ddots & \vdots     \\
    #2_{#31}    & #2_{#32}    & \cdots & #2_{#3#4}
  \end{pmatrix}
  \]
}

% 简化版命令(默认矩阵名为A,元素符号为a): \quickmatrix{行数}{列数}
\newcommand{\quickmatrix}[2]{\flexmatrix{A}{a}{#1}{#2}}

\begin{document}
\title{基础知识}
\author{张志聪}
\maketitle

% \begin{zremark}
%   1.3.2 定理1(零空间定理)的证明中:
%   \begin{align*}
%     \left(\sum\limits_{i = 0}^m c_i E^i\right) u 
%     = \sum\limits_{i = 0}^m c_i (E^iu) 
%     = \sum\limits_{i = 0}^m c_i \lambda^i u
%     = p(\lambda) u
%   \end{align*}
%   每一步具体是如何得到。
% \end{zremark}

% \textbf{证明:}

% \begin{align*}
%     \left(\sum\limits_{i = 0}^m c_i E^i\right) u 
%     = \sum\limits_{i = 0}^m c_i (E^i u) 
% \end{align*}
% 这一步的展开方式,完全是数学定义规定的,只是书中没有提而已;

% \begin{align*}
%   \sum\limits_{i = 0}^m c_i (E^iu) 
%   = \sum\limits_{i = 0}^m c_i \lambda^i u
% \end{align*}
% 书中有说明理由;

% \begin{align*}
%   \sum\limits_{i = 0}^m c_i \lambda^i u
%   = p(\lambda) u
% \end{align*}
% 这一步是高等代数中“向量的数乘 + 加法的运算规则”的运用。

% \begin{zremark}
%   谱定理的证明。
% \end{zremark}

\begin{zremark}
  矩阵范数$\|\cdot\|_F$满足相容性,其中$A = (a_{ij})$是$n$阶矩阵。
\end{zremark}

\textbf{证明:}

对任意$A \in M_{m \times l}, B \in M_{l \times n}$,并利用复数的模满足:
\begin{align*}
  |x + y| \leq |x| + |y|
\end{align*}
我们有
\begin{align*}
  \|A B\|_F
  & = \left(\sum \limits_{i = 1}^m \sum \limits_{j = 1}^n |a_{i1}b_{1j} + a_{i2}b_{2j} + \cdots + a_{il}b_{lj}|^2 \right)^{\frac{1}{2}} \\
  & \leq \left(\sum \limits_{i = 1}^m \sum \limits_{j = 1}^n (|a_{i1}b_{1j}| + |a_{i2}b_{2j}| + \cdots + |a_{il}b_{lj}|)^2 \right)^{\frac{1}{2}} 
\end{align*}
利用柯西-施瓦茨不等式,我们有
\begin{align*}
  & \left(\sum \limits_{i = 1}^m \sum \limits_{j = 1}^n (|a_{i1}b_{1j}| + |a_{i2}b_{2j}| + \cdots + |a_{il}b_{lj}|)^2 \right)^{\frac{1}{2}} \\
  & \leq \left(\sum \limits_{i = 1}^m \sum \limits_{j = 1}^n (|a_{i1}|^2 + |a_{i2}|^2 + \cdots + |a_{il}|^2)(|b_{1j}|^2 + |b_{2j}|^2 + \cdots + |b_{lj}|^2) \right)^{\frac{1}{2}} \\
  & = \left(\sum \limits_{i = 1}^m \sum \limits_{k = 1}^l |a_{ik}|^2\right)^{\frac{1}{2}} \left(\sum \limits_{k = 1}^l \sum \limits_{j = 1}^n |b_{kj}|^2\right)^{\frac{1}{2}} \\
  & = \|A\|_F \|B\|_F
\end{align*}
命题得证。



\end{document}