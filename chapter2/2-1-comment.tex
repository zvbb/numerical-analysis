\documentclass{article}
\usepackage{mathtools} 
\usepackage{fontspec}
\usepackage[UTF8]{ctex}
\usepackage{amsthm}
\usepackage{mdframed}
\usepackage{xcolor}
\usepackage{amssymb}
\usepackage{amsmath}


% 定义新的带灰色背景的说明环境 zremark
\newmdtheoremenv[
  backgroundcolor=gray!10,
  % 边框与背景一致,边框线会消失
  linecolor=gray!10
]{zremark}{说明}

% 通用矩阵命令: \flexmatrix{矩阵名}{元素符号}{行数}{列数}
\newcommand{\flexmatrix}[4]{
  \[
  #1 = \begin{pmatrix}
    #2_{11}     & #2_{12}     & \cdots & #2_{1#4}   \\
    #2_{21}     & #2_{22}     & \cdots & #2_{2#4}   \\
    \vdots      & \vdots      & \ddots & \vdots     \\
    #2_{#31}    & #2_{#32}    & \cdots & #2_{#3#4}
  \end{pmatrix}
  \]
}

% 简化版命令(默认矩阵名为A,元素符号为a): \quickmatrix{行数}{列数}
\newcommand{\quickmatrix}[2]{\flexmatrix{A}{a}{#1}{#2}}

\begin{document}
\title{2.1}
\author{张志聪}
\maketitle

\begin{zremark}
  定理1(加法的相对舍入误差定理) 证明过程中
  \begin{align*}
    & \{[S_k(1 + \rho_k) + x_{k + 1}](1 + \delta_k) - (S_k + x_{k + 1})\} / S_{k + 1} \\
    & = \delta_k + \rho_k(S_k / S_{k + 1}) (1 + \delta_k)
  \end{align*}
\end{zremark}

\textbf{证明:}
  \begin{align*}
    & \{[S_k(1 + \rho_k) + x_{k + 1}](1 + \delta_k) - (S_k + x_{k + 1})\} / S_{k + 1} \\
    & = \{ (S_k + S_k \rho_k + x_{k + 1}) (1 + \delta_k) - S_k - x_{k + 1} \} / S_{k + 1} \\
    & = \{ S_k + S_k \rho_k + x_{k + 1} + S_k \delta_k + S_k \rho_k \delta_k + x_{k + 1} \delta_k - S_k - x_{k + 1} \} / S_{k + 1} \\
    & = \{ S_k \rho_k + S_k \delta_k + S_k \rho_k \delta_k + x_{k + 1} \delta_k \} / S_{k + 1} \\
    & = \{ \delta_k(S_k + x_{k + 1}) + \rho_k S_k  ( 1 + \delta_k) \} / S_{k + 1} \\
    & = \{ \delta_k S_{k+1} + \rho_k S_k  ( 1 + \delta_k) \} / S_{k + 1} \\
    & = \delta_k + \rho_k(S_k / S_{k + 1}) (1 + \delta_k)
  \end{align*}

\begin{zremark}
  例3中“小数点后面第m个位是正确的”数学定义(绝对误差限与精确到小数点后$m$位)):

  近似值$\tilde{x}$精确到小数点后第$m$位(或说第$m$位小数是正确的),如果其绝对误差
  \begin{align*}
    |x - \tilde{x}| < 0.5 \times 10^{-m}
  \end{align*}
  此处$x$是精确值。这意味着如果我们要把精确值$x$\textbf{四舍五入}到小数点后$m$位,结果必定与
  $\tilde{x}$\textbf{四舍五入}到小数点后$m$位一致。
\end{zremark}


\end{document}